\documentclass{tufte-handout}
\setcounter{secnumdepth}{1}
\usepackage{amsmath}
\newtheorem{app}{Application}
\newtheorem{example}{Example}
\newtheorem{exercise}{Exercise}
%\geometry{showframe}% for debugging purposes -- displays the margins
% all kinds of defaults from the template provided by overleaf
\usepackage{booktabs}
\usepackage{units}
\usepackage{fancyvrb}\fvset{fontsize=\normalsize}
\usepackage{multicol}
\usepackage{graphicx}
\setkeys{Gin}{width=\linewidth,totalheight=\textheight,keepaspectratio}
\graphicspath{{graphics/}}

% my own math shortcuts
\newcommand{\re}{\mathbb{R}}


\title{Notes on stuff}
\author[Diego Navarro}
\date{}  


\begin{document}

\maketitle% this prints the handout title, author, and date

\begin{abstract}
\noindent This documents my ongoing understanding on simplicial complexes, Hodge Laplacians and parts of Hodge theory. Please expect it to contain inaccuracies\sidenote{Visual layout has rhetorical value, and the use of \LaTeX in particular conveys a kind of rigor and certainty that's not always there -- at least not in my case. Indeed the very choice of authoring language and layout tooling can be read as pedantry, but I'd prefer you to think of it as aspirational.}, errors and downright misapprehensions, and leave me feedback whenever appropriate. 
\end{abstract}

%\printclassoptions
\section{Simplices in real spaces}

\subsection{The standard 1-simplex}
The standard 1-simplex\sidenote{`Simplices' is the plural form of `simplex': one simplex, two simplices, three simplices. I will forever be bothered by the fact that `complex' doesn't follow this pattern.} $\s_1$ is the set of pairs of positive real numbers $(x_1,x_2)$ such that $x_1+x_2=1$. My intuition for this is that a simplex is a ``budget'': the space of possible ways to allocate resources to two buckets. 

Note that $s_1$ is one-dimensional despite being a subset of $\re^2$.  Since $x_1$ alone determines a point $(x_1,1-x_1)$ in the simplex, $s_1$ can be identified with the unit line segment $[0,1]$. To recover a two-dimensional picture, it's sufficient to project this line segment to either axis.
\subsection{The standard 2-simplex}
Likewise, the standard 2-simplex $s_2$ is the set of triples of positive real numbers $(x_1,x_2,x_3)$ such that $x_1+x_2+x_3=1$. This is clearly two-dimensional, as it can be projected on any pair of coordinates (either $(x_1,x_2)$ or $(x_1,x_3)$ or $(x_2,x_3)$. What does it look like as a subset of $\re^3? 

If we hold $x_3$ constant at, for example, $0.1$, then we have that $x_1+x_2=0.9$. 
Then $s_2$ can be viewed as the union of `scaled segments' for every possible value along the third coordinate:

\begin{equation}
    s_2 = \bigcup_{\alpha\in[0,1]} \{x_1+x_2 = 1-\alpha\} 
\end{equation}

the union of line segments interpolating points between $(0,\alpha,1-\alpha)$ and $(\alpha,0,1-\alpha)$. Looking at the first two coordinates only, this looks like a triangle bounded by $(0,1),(1,0)$ and the diagonal line between these vertices. By trilateral symmetry, this is also true for the other two pairs of coordinates; therefore $s_2$ must look like a (filled) triangle from every orthogonal perspective. Now, much like $s_1$ ca be identified with any of its two 1-dimensional projections, $s_2$ can be identified with an equilateral triangle\sidenote{This is actually exploited in data visualization to represent objects that have ingredients (like the NPK - nitrogen, phosphorus, potassium - composition of soils)}.
\section{General $n$-simplices}
More generally, the standard $n$-simplex is the subset of $\re^{n+1}$ such that
\begin{equation}
    s_n = x_1 + \cdots + x_{n+1} = 1, \quad \quad \quad\quad x_1, \cdots, x_{n+1} > 0
\end{equation}

Once again, this can be seen as the union

\begin{equation}
  s_n =  \bigcup_{\alpha \in [0,1]} \{x_1+\cdots+x_n = 1-\alpha\ |\ x_1,\ldots,x_n>0\}
\end{equation}
each term of which is a scaled-down version of $s_{n-1}$ until it becomes a single point (a vertex). As in the $n=2$ case, by $(n+1)$-lateral symmetry this happens in $n+1$ directions so there's $(n+1)$ vertices.
\begin{exercise}
What does a $3$-simplex look like? Can you visualize a $4$-simplex animating over one of the dimensions as ``time''?
\end{exercise}

For additional generality we can let simplices lie anywhere in real $n$-space; for example, a nonstandard 1-simplex will interpolate between two arbitrary vectors:

\begin{equation}
    s_{n}(u_1,\ldots,u_n) = \left\{\sum \theta_i u_i = 1\ |\  \theta_i>0; u_i\in \re^n \right\}
\end{equation}

As noted, standard simplices naturally represent data about composition (how much of a cake is saturated fats and sugars and so on). They can also represent the space of possible discrete probability distributions with $n$ outcomes: a binary random variable like a (possibly rigged) coin toss is usually specified by a single value $\text{Prob}(\text{heads})$, but this is because $\text{Prob}(\text{heads})+\text{Prob}(\text{tails})=1$. In other words, the space of all possible (in the general case rigged) coins\sidenote{A coin is a random variable.} is just $s_2$. 

\begin{app}{Dirichlet priors} -- A particular coin (one that you can find in your trouser pockets) is an element of the 2-simples $s_2$. But how can we determine which element it is? One tactic is to throw it an infinite number of times until (by virtue of the Law of Large Numbers\sidenote{Exactly what it seems.}) it become a known quantity. Under more realistic circumstances, Bayesian statistics allows us to represent imperfect knowledge or uncertainty. The ``natural'' prior in this case is the Beta distribution, which takes values in $[0,1]=s_2$. More generally, the Dirichlet distribution is the prior distribution for a discrete random variable with arbitary $n$ outcomes. The range of possible values for such a Dirichlet is the $n$-simplex; in other words, sampling from this distribution generates points in the $n$-simplex.
\end{app}
\begin{app}{Aitchison geometry} -- Simplices are not vector spaces under Euclidean geometry: $u=(0.8,0.1,0.1), v = (0.6,0.2,0.2)$ are both elements of $s_3$, but $u+v$ is not. But the standard operations $+,\cdot$ can be replaced by alternate $\oplus,\odot$ so that the vector space laws apply:
\[
x\oplus y = \begin{pmatrix}\frac{x_1 y_1}{\sum x_i y_i} & \cdots & \frac{x_n y_n}{\sum x_i y_i}\end{pmatrix}
\]
\[
\alpha \odot x = \begin{pmatrix}\frac{x_1^\alpha}{\sum x_i^\alpha} & \cdots & \frac{x_n^\alpha}{\sum x_i^\alpha} \end{pmatrix}
\]
\end{app}
\section{Abstract simplices}

\section{Simplicial complexes}

\section{Homology, cohomology}

\section{Connections with Hodge theory}
\nocite{*}
\bibliography{refs}
\bibliographystyle{siam}



\end{document}
