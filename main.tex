\documentclass{tufte-handout}
\setcounter{secnumdepth}{1}
\usepackage{amsmath}
\newtheorem{app}{Application}
\newtheorem{example}{Example}
\newtheorem{exercise}{Exercise}
\newcommand{\define}{\textsc}
%\geometry{showframe}% for debugging purposes -- displays the margins
% all kinds of defaults from the template provided by overleaf
\usepackage{booktabs}
\usepackage{units}
\usepackage{fancyvrb}\fvset{fontsize=\normalsize}
\usepackage{multicol}
\usepackage{graphicx}
\setkeys{Gin}{width=\linewidth,totalheight=\textheight,keepaspectratio}
\graphicspath{{graphics/}}

% my own math shortcuts
\newcommand{\re}{\mathbb{R}}


\title{Notes on stuff}
\author[Diego Navarro}
\date{}  


\begin{document}

\maketitle% this prints the handout title, author, and date

\begin{abstract}
\noindent This documents my ongoing understanding on simplicial complexes, Hodge Laplacians and parts of Hodge theory. Please expect it to contain inaccuracies\sidenote{Visual layout has rhetorical value, and the use of \LaTeX in particular conveys a kind of rigor and certainty that's not always there -- at least not in my case. Indeed the very choice of authoring language and layout tooling can be read as pedantry, but I'd prefer you to think of it as aspirational.}, errors and downright misapprehensions, and leave me feedback whenever appropriate. 
\end{abstract}

%\printclassoptions
\section{Simplices in real spaces}

\subsection{The standard 1-simplex}
The standard 0-simplex is a real number $x_1\in[0,1]$.

The standard 1-simplex\marginnote{`Simplices' is the plural form of `simplex': one simplex, two simplices, three simplices.} $S_1$ is the set of pairs of positive real numbers $(x_1,x_2)$ such that $x_1+x_2=1$. If the pair $(x_1, x_2)$ is seen as an element of $\re^2$, this means the 1-simplex is the segment of the graph for $y=1-x$ where both $x,y$ are positive. A more verbose definition follows:
\begin{equation}
    S_1 := \{(x_1,x_2)\ |\ x_1+x_2=1, x_1>0, x_2>0\}
\end{equation}

\marginnote{My intuition for the 1-simplex is a ``budget'': the space of possible ways to allocate resources to two buckets} From this equation it follows that $0\leq x_1, x_2\leq 1$. Then we can equivalently define 
\begin{equation}
    S_1 = \bigcup_{\alpha\in [0,1]} \{{(x_1,\alpha)}\ |\ x_1 = 1-\alpha]\}
\label{union-1}
\end{equation}
This definition makes it clearer that, despite being originally identified as a pair of numbers, $S_1$ is 1-dimensional and bijectively identified with the unit interval $[0,1]$. 

Note that the symmetry in $x_1+x_2=1$ means we could have swapped the order (or labels) of summands and taking unions over pairs like $(\alpha,x_2)$ would yield the same set. Visually we can imagine an animation that sweeps the vertical axis and records the points where $x=1-y$, or one that sweeps over the horizontal axis and records the points where $y=1-x$.\marginnote{This is overthinking for the 1-simplex but helps me see the 2- and specially the 3-simplex.}

Note, finally, that the issue of length is more or less arbitrary: we could easily define scaled $1$-simplices corresponding to line segments of length $c>0$
\[
c * S_1 = \{x_1 + x_2 = c\ |\ x_1,x_2>0\}
\]
This is basically useful only to visualize the 2- and 3-simplex, so don't hang on to it too tightly.

\subsection{The standard 2-simplex}
The standard 2-simplex $S_2$ is the set of triples of positive real numbers $(x_1,x_2,x_3)$ such that $x_1+x_2+x_3=1$. Choosing the the third coordinate to range over, this means

\begin{equation}
    S_2 =  \bigcup_{\alpha\in[0,1]} \{(x_1,x_2,\alpha)\ |\ (x_1,x_2)\in (1-\alpha)*S_1\}
    \label{union-view-2}
\end{equation}
This is a union of line segments with lengths ranging from from an unit segment to a single point at $(0,0,1)$. We call the latter a \define{vertex}. Again, by trilateral symmetry $(1,0,0)$ and $(0,1,0)$ must also be vertices; and since we could have chosen to taken unions over pairs like $(x_1, \alpha,x_3)$ or $(\alpha,x_2,x_3)$, two additional unit segments must exist. If the 2-simplex is seen as embedded in $\re^3$, these must correspond to the line segments (\define{edges})
\begin{align*}
\{(x_1,1-x_1,0)\ &|\  x_1\in[0,1]\}\\
\{(0,x_2,1-x_2)\ &|\ x_2 \in[0,1]\}\\
\{(1-x_3,0,x_3)\ &|\ x_3\in[0,1]\}
\end{align*}

each of which is a 3D rotation of the 1-simplex $S_1$ to make the vertices touch. Again by trilateral symmetry, to each of these edges corresponds a range of shrinking segments filling the space until the next vertex. ``Forgetting'' any one of these three coordinates, we project a triangle in one of the three planes that intersect at the origin. This triangle must be equilateral\sidenote{This is actually exploited in data visualization to represent objects that have ingredients (like the NPK - nitrogen, phosphorus, potassium - composition of soils)}, since all edges are unit-length.
\subsection{n-simplices}
Generalizing the reasoning above, the standard $n$-simplex is the following $\re^{n+1}$:
\begin{equation}
    S_n = \{x_1 + \cdots + x_{n+1} = 1\ |\  x_1, \cdots, x_{n+1} \geq 0 \}the 
    \label{standard-simplex}
\end{equation}

Once again, this can be seen as the union

\begin{equation}
  S_n =  \bigcup_{\alpha \in [0,1]} \{x_1+\cdots+x_n = 1-\alpha\ |\ x_1,\ldots,x_n>0\},
\end{equation}
and since each element is a scaled $(n-1)$-simplex, this is again equivalent to 
\begin{equation}
    S_n = \bigcup_{\alpha\in[0,1]} \{(x_1,\cdots,x_n,\alpha) | \ \{(x_1,\cdots,x_n,\alpha)\in (1-\alpha)*S_{n-1}  \}
\end{equation}
\begin{exercise}
What does a $3$-simplex look like?
\end{exercise}
\subsection{General $n$-simplices}
For additional generality we can let simplices lie anywhere in real $n$-space as follows:

\begin{equation}
    S_{n}(u_1,\ldots,u_{n+1}) = \left\{\sum_{i=1}^{n+1} \theta_i u_i = 1\ |\  \theta_i>0; u_i\in \re^{n+1} \right\}
    \label{geometric-simplex}
\end{equation}


where $u_1,\ldots,u_{n+1}$ are the \emph{vertices}. The standard simplices are, of course, general simplices whose vertices are the canonical base vectors $e_1=(1,0,\cdots),e_2=(0,1,0,\cdots),\ldots$. This is easier to see if we let $\Theta=(\theta_i), U = (u_{i,.})$  be a vector of coefficients ans a matrix of vertices as columns so that

\begin{equation}
    U\Theta= 1
\end{equation}
In the case where $U$ is the matrix with the canonical base vectors, this is clearly true and $\theta_1,\cdots$ take the role $x_1, \cdots$ had previously.
Furthermore, the vertices must in the general case be linearly independent; thus $U^{-1}$ exists and takes the general simplex to its standard counterpart.
% As noted, standard simplices naturally represent data about composition (how much of a cake is saturated fats and sugars and so on). They can also represent the space of possible discrete probability distributions with $n$ outcomes: a binary random variable like a (possibly rigged) coin toss is usually specified by a single value $\text{Prob}(\text{heads})$, but this is because $\text{Prob}(\text{heads})+\text{Prob}(\text{tails})=1$. In other words, the space of all possible (in the general case rigged) coins\sidenote{A coin is a random variable.} is just $S_2$. 

% \begin{app}{Dirichlet priors} -- A particular coin (one that you can find in your trouser pockets) is an element of the 2-simplex $S_2$. The natural Bayesian prior in this case is the Beta distribution, which takes values in $[0,1]=S_2$. More generally, the Dirichlet distribution is the prior distribution for a discrete random variable with arbitary $n$ outcomes. The range of possible values for such a Dirichlet is the $n$-simplex; in other words, sampling from this distribution generates points in the $n$-simplex.
% \end{app}
% \begin{app}{Aitchison geometry} -- Simplices are not vector spaces under Euclidean geometry: $u=(0.8,0.1,0.1), v = (0.6,0.2,0.2)$ are both elements of $S_3$, but $u+v$ is not. But the standard operations $+,\cdot$ can be replaced by alternate $\oplus,\odot$ so that the vector space laws apply:
% \[
% x\oplus y = \begin{bmatrix}{x_1 y_1}/{\sum x_i y_i} & \cdots & {x_n y_n}/{\sum x_i y_i}\end{bmatrix}
% \]
% \[
% \alpha \odot x = \begin{bmatrix}{x_1^\alpha}/{\sum x_i^\alpha} & \cdots & {x_n^\alpha}/{\sum x_i^\alpha} \end{bmatrix}
% \]
% \end{app}
 
\section{Simplicial complexes}
\subsection{Abstract simplices}
Given the definition (\ref{geometric-simplex}) $n$-simplex can be fully be specified\sidenote{But not fully \emph{identified}; a triangle is still distinct from its vertices or even its edges.} by its $(n+1)$ vertices, each a vector of $\re^{n+1}$. 
\section{Homology, cohomology}

\section{Connections with Hodge theory}
\nocite{*}
\bibliography{refs}
\bibliographystyle{siam}



\end{document}
