\documentclass{tufte-handout}

%\geometry{showframe}% for debugging purposes -- displays the margins

\usepackage{amsmath}

% Set up the images/graphics package
\usepackage{graphicx}
\setkeys{Gin}{width=\linewidth,totalheight=\textheight,keepaspectratio}
\graphicspath{{graphics/}}

\title{Notes on simplicial complex and differential forms}
\author[Diego Navarro}
\date{}  % if the \date{} command is left out, the current date will be used

% The following package makes prettier tables.  We're all about the bling!
\usepackage{booktabs}

% The units package provides nice, non-stacked fractions and better spacing
% for units.
\usepackage{units}

% The fancyvrb package lets us customize the formatting of verbatim
% environments.  We use a slightly smaller font.
\usepackage{fancyvrb}
\fvset{fontsize=\normalsize}

% Small sections of multiple columns
\usepackage{multicol}

% Provides paragraphs of dummy text
\usepackage{lipsum}

% These commands are used to pretty-print LaTeX commands
\newcommand{\doccmd}[1]{\texttt{\textbackslash#1}}% command name -- adds backslash automatically
\newcommand{\docopt}[1]{\ensuremath{\langle}\textrm{\textit{#1}}\ensuremath{\rangle}}% optional command argument
\newcommand{\docarg}[1]{\textrm{\textit{#1}}}% (required) command argument
\newenvironment{docspec}{\begin{quote}\noindent}{\end{quote}}% command specification environment
\newcommand{\docenv}[1]{\textsf{#1}}% environment name
\newcommand{\docpkg}[1]{\texttt{#1}}% package name
\newcommand{\doccls}[1]{\texttt{#1}}% document class name
\newcommand{\docclsopt}[1]{\texttt{#1}}% document class option name

\begin{document}

\maketitle% this prints the handout title, author, and date

\begin{abstract}
\noindent This documents my ongoing understanding on simplicial complexes, Hodge Laplacians and parts of Hodge theory. Please expect it to contain inaccuracies\sidenote{Visual layout has rhetorical value, and the use of \LaTeX in particular conveys a kind of rigor and certainty that's not always there -- at least not in my case. Indeed the very choice of authoring language and layout tooling can be read as pedantry, but I'd prefer you to think of it as aspirational.}, errors and downright misapprehensions, and leave me feedback whenever appropriate. 
\end{abstract}

%\printclassoptions
\section{Simplices}

\section{Simplicial complexes}

\section{Homology, cohomology}

\section{Connections with Hodge theory}
\nocite{*}
\bibliography{refs}
\bibliographystyle{siam}



\end{document}
